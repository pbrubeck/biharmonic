%!TEX root = stokes_paper.tex

\section{Introduction}

In this paper we consider numerical solutions to the biharmonic BVP
\begin{equation} \label{eq:bvp}
\nabla^4 \psi = 0, \quad \text{in }\Omega,
\end{equation}
with appropriate boundary conditions on $\partial\Omega$, imposed on $\psi$ (Dirichlet) or its derivatives (Neumann).


Finding solutions to Stokes flow presents challenges in spite of the fact that the equations involved are linear. In most common applications, one encounters geometries with corners, and discontinuous boundary conditions, which result in point singularities in the solution. 


\endinput
\begin{itemize}
\item Discuss regularity in the physical variables $u,v,p,\omega,\psi$. 
\item We have harmonic functions $p,\omega$, which belong to $H^0(\Omega)$
\item We also have biharmonic functions $\psi,u,v$, which belong to $H^2(\Omega)$
\end{itemize}
%!TEX root = stokes_paper.tex

\section{Unbounded domains}
The nonuniqueness of the Goursat representation becomes a relevant concern when dealing with unbounded domains, e.g. an infinitely long pipe or channel. One sensible approach is to decompose $\psi = \psi_\infty + \psi_0$, where $\psi_\infty$ is prescribed analytically such that is exponentially satisfies the Dirichlet boundary conditions as $z\to\infty$, so the numerical task is to approximate $\psi_0$ such that it satisfies the modified boundary conditions near the origin and exponentially decays to 0 as $z\to\infty$. 

One may choose to use basis functions for $f(z)$ and $g(z)$ that go to $0$ as $z\to\infty$, in the hope that they decay like $\psi_0$, e.g. rational functions with simple poles placed outside $\Omega$. The nonuniqueness of the Goursat representation might interfere with the accuracy of such approximation. Recall that $\psi$ is invariant under the transformation
\begin{equation}
f(z) \to f(z) + \lambda z + C, \quad g(z) \to g(z) + \conj{C}z + \alpha.
\end{equation}
Hence, in order to ensure the decay on both analytic functions, one must consider four additional constraints in the least-squares problem. One way is fixing the Airy stress function at three non-colinear points inside $\Omega$, that will certainly set the constant term of $f(z)$ to zero and will make the linear terms in $g(z)$ vanish. The missing parameter, $\lambda$ can be controlled by setting the pressure to zero at one point inside $\Omega$.
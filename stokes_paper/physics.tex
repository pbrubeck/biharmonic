%!TEX root = stokes_paper.tex

\section{Physical interpretation: Stokes flow}

Let $\vec{u}=(u,v)^T$ be the velocity field of a steady incompressible 2D fluid flow, and let $p$ be the associated pressure field, which is defined up to a constant. In the limit $\text{Re}\to 0$ and in the absence of body forces, the steady-state equations for the conservation of momentum and the incompressibility constraint are given by
\begin{subequations}\label{eq:stokes}
\begin{align}
-\nabla^2 \vec{u} + \nabla p &= 0,\label{eq:momen}\\
\nabla\cdot\vec{u} &= 0. \label{eq:incomp}
\end{align}
\end{subequations}
A common solution technique is to introduce the stream function $\psi$ by letting
\begin{equation}
u=\frac{\partial\psi}{\partial y}, \quad 
v=-\frac{\partial\psi}{\partial x}.
\end{equation}
Then, it is easy to verify that \eqref{eq:incomp} is automatically satisfied. Now define the vorticity as
\begin{equation}
\omega := \frac{\partial v}{\partial x} - \frac{\partial u}{\partial y}  = -\nabla^2 \psi.
\end{equation}
Note that \eqref{eq:momen} takes the form of the Cauchy-Riemann conditions 
\begin{equation}\label{eq:pomega}
-\nabla^2u=\frac{\partial\omega}{\partial y} = -\frac{\partial p}{\partial x} , \quad 
-\nabla^2v=-\frac{\partial\omega}{\partial x} = -\frac{\partial p}{\partial y},
\end{equation}
which directly imply that $p$ and $\omega$ are harmonic conjugates. Therefore, the stream function-vorticity formulation
\begin{subequations}
\begin{align}
-\nabla^2\psi &= \omega,\label{eq:psi}\\
-\nabla^2\omega &= 0,\label{eq:omega}
\end{align}
\end{subequations}
is equivalent to the system of equations \eqref{eq:stokes}. Moreover, one can take the Laplacian on both sides of \eqref{eq:psi} and use \eqref{eq:omega} to reveal that $\psi$ is biharmonic. Thus we have reduced the original system of three equations in three unknowns to a biharmonic equation. By Theorem \ref{th:goursat}, one can write
\begin{equation}
\psi = \frac{1}{2i} \left[g(z) - \conj{g(z)} + \conj{z}f(z) - z\conj{f(z)}\right].
\end{equation}
so that the velocity components are given in terms of $f(z)$ and $g(z)$,
\begin{equation}
u-iv = 2i\frac{\partial \psi}{\partial z} = g'(z) + \conj{z} f'(z) - \conj{f(z)}.
\end{equation}
Since $\psi$ is invariant under the transformations of Theorem \ref{th:goursat}, so are $u$ and $v$. The vorticity is then given by
\begin{equation}
-\omega = \nabla^2 \psi =4\frac{\partial^2\psi}{\partial \conj{z}\partial z} = -2i \left( f'(z) - \conj{f'(z)}\right) = 4\Im\left\{f'(z)\right\},
\end{equation}
and from \eqref{eq:pomega} we deduce that
\begin{equation}
p-i\omega = 4 f'(z).
\end{equation}
Since the pressure is uniquely defined up to a constant, we can relate this freedom to a pressure shift of $p_0=4\lambda$. 


Let us further introduce the Airy stress function $A$, which we will define as the biharmonic conjugate of $-2\psi$, implying that $-2(p-i\omega) = \nabla^2 (A-2i\psi)$


a potential for the stress tensor $\sigma_{ij} = u_{i,j} + u_{j,i} -p \delta_{ij}$,
\begin{equation}
\matlabmatrix{\sigma_{11} \sigma_{12}; \sigma_{21} \sigma_{22}} = 
\matlabmatrix{A_{yy} -A_{xy}; -A_{yx} A_{xx}}.
\end{equation}
Since only second derivatives of $A$ have a physical interpretation, $A$ is uniquely defined up to a linear term. We can connect the three corresponding degrees of freedom with those on the arbitrariness on the Goursat representation, defined as the biharmonic conjugate of $\psi$,
\begin{equation}\label{eq:stress}
-\frac{A}{2}+i\psi := \conj{z}f(z) + g(z).
\end{equation}
We note that \eqref{eq:stress} can be transformed into
\begin{equation}
-\frac{A}{2}+i\psi = \conj{z}f(z) + g(z) + \left(\lambda \conj{z}z + C\conj{z} + \conj{C}z + \alpha \right), 
\end{equation}
without changing the physical description of the fluid. From the terms in parenthesis, we observe that only $\lambda \conj{z}z$ has an effect on the stress tensor, namely the pressure shift. The rest of these terms are of the form $c_1 x+c_2 y+\alpha$ which coincides with the arbitrariness in the definition of $A$, and correspond to rigid-body displacements.


\subsection{Moffatt eddies}

The asymptotic solution ($r \ll r_0$) for the anti-symmetric flow inside a wedge of angle $2\alpha$ caused by a perturbation at infinity was studied by Moffat in \cite{Moffatt64},
\begin{equation}
\psi_\lambda(r,\theta) \sim \Re\left\{\left(\frac{r}{r_0}\right)^\lambda \left[\cos{(\lambda\alpha)}\cos{\left((\lambda-2)\theta\right)}+\cos{\left((\lambda-2)\alpha\right)}\cos{(\lambda\theta)}\right]\right\},
\end{equation}
where, in order to satisfy the no-slip conditions $\psi_\lambda=\uvec{n}\cdot\nabla\psi_\lambda=0$ at $\theta=\pm\alpha$, the parameter $\lambda$ must be a root of
\begin{equation}\label{eq:roots}
\sin{\left(2\alpha(\lambda-1)\right)} + (\lambda-1)\sin{(2\alpha)}=0.
\end{equation}
If $2\alpha>146^\circ$, the non-trivial roots of \eqref{eq:roots} are complex, $\lambda=p + iq$, which means that $\psi_\lambda$ presents radial oscillations (periodic in $\log(r)$) of decaying amplitude at a rate of $r^{p}$, which is manifested as a sequence of counter-rotating vortices, commonly known as Moffat eddies.


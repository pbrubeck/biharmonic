%!TEX root = ../main.tex
\usepackage[english]{babel}
\usepackage[utf8]{inputenc}

\usepackage{./tex/matlabmatrix}
\usepackage{embedfile}
\embedfile{\jobname.tex}

\usepackage{physics}
\usepackage{array}
\usepackage{empheq}
\usepackage[export]{adjustbox}
\usepackage[most]{tcolorbox}

% AMSLaTeX packages
\usepackage{amsthm}
\usepackage{amsmath}
\usepackage{amsfonts}
\usepackage[algoruled]{algorithm2e}

\usetheme{default}
\useoutertheme{default}
% we want to use images
%\usepackage{graphicx}
\usepackage{graphbox} %loads graphicx package
\usepackage{movie15}
\usepackage{hyperref}
\usepackage{subcaption}


% table relates packages
\usepackage{booktabs}
\usepackage{multicol}
\usepackage{multirow}
% pick a font
\usepackage{palatino}           
% \usepackage{times}
\usepackage{tikz}
\usetikzlibrary[positioning,arrows,decorations.pathmorphing,backgrounds,fit,calc]
% \AtBeginSection[]  % "Beamer, do the following at the start of every section"
% {
%   \begin{frame}<beamer> 
%     \frametitle{Outline} % make a frame titled "Outline"
%     \tableofcontents[currentsection]  % show TOC and highlight current section
%   \end{frame}                    
% }

% \AtBeginSubsection[]
% {
%   \begin{frame}
%     \frametitle{Outline}
%     \tableofcontents[currentsection,currentsubsection]
%   \end{frame}
% }

\AtBeginSection[]
{
   \begin{frame}
       \frametitle{Outline}
       \tableofcontents[currentsection]
   \end{frame}
}

\newcommand{\ebox}[1][1em]{\framebox[#1]{\phantom{M}}}

\setlength\arraycolsep{1.4pt}% some length

%gets rid of navigation symbols
\setbeamertemplate{navigation symbols}{}

%gets rid of bottom navigation bars
\setbeamertemplate{footline}[page number]{}
\setbeamertemplate{headline}{}

\renewcommand\vec{\mathbf}
\newcommand{\uvec}[1]{\hat{\mathbf{#1}}}
\newcommand{\conj}[1]{{\overline{#1}}}


\DeclareMathOperator{\diag}{diag}
\DeclareMathOperator{\blkdiag}{blockdiag}
\DeclareMathOperator{\vecop}{vec}
\newcommand{\kron}{\otimes}
\newcommand{\transp}{\text{T}}
\newcommand{\lavec}[1]{\underline{#1}}
\newcommand{\phvec}[1]{\mathbf{#1}}
\newcommand{\bigo}{\mathcal{O}}

\newcommand{\core}{\mathcal{S}(\mathbf{r})}
\renewcommand{\prec}{P^{[\mathbf{r}]}}

\newtcbox{\redbox}[1][]{%
	nobeforeafter, math upper, tcbox raise base,
	enhanced, colframe=red!30!black,
	colback=red!30, boxrule=1pt,
	#1}

\newtcbox{\bluebox}[1][]{%
	nobeforeafter, math upper, tcbox raise base,
	enhanced, colframe=blue!30!black,
	colback=blue!30, boxrule=1pt,
	#1}

\newtcbox{\greenbox}[1][]{%
	nobeforeafter, math upper, tcbox raise base,
	enhanced, colframe=green!30!black,
	colback=green!30, boxrule=1pt,
	#1}

\newtcbox{\yellowbox}[1][]{%
	nobeforeafter, math upper, tcbox raise base,
	enhanced, colframe=yellow!30!black,
	colback=yellow!30, boxrule=1pt,
	#1}

\makeatletter
\newbox\@backgroundblock
\newenvironment{backgroundblock}[2]{%
	\global\setbox\@backgroundblock=\vbox\bgroup%
	\unvbox\@backgroundblock%
	\vbox to0pt\bgroup\vskip#2\hbox to0pt\bgroup\hskip#1\relax%
}{\egroup\egroup\egroup}
\addtobeamertemplate{background}{\box\@backgroundblock}{}
\makeatother


\addtobeamertemplate{navigation symbols}{}{%
	\usebeamerfont{footline}%
	\usebeamercolor[bg]{footline}%
	\hspace{1em}%
	\textcolor{\black}{\insertframenumber/\inserttotalframenumber}
}


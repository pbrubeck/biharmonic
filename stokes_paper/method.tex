%!TEX root = stokes_paper.tex

\section{Numerical Method \label{sec:method}}
In this section we describe the lightning Stokes solver that solves \eqref{eq:bih}--\eqref{eq:bcs} when $\Omega$ is a curved polygon with $K$ corners $\{w_k\}_{k=1}^K$. The method consists of the rational approximation of the two analytic functions $f(z)$ and $g(z)$ that appear in the Goursat representation. We choose $N$ basis functions $\{\psi_j(z)\}_{j=1}^N$ with fixed poles, and we determine the coefficients by approximately solving a linear least-squares problem in $M$ sample boundary points $\{z_i\}_{i=1}^M \subset \partial\Omega$. We increase $N$ adaptively until the residual is below a certain tolerance or the refinement does not bring any improvement. Our implementation is based on the publicly available code $\texttt{laplace.m}$ \cite{tref20}, currently in version 6, and it provides a benchmark of practical implementation and speed of the algorithm.


\subsection{Rational functions}
Following \cite{gopal19}, we use rational functions to approximate the analytic functions appearing in the
Goursat representation, which guarantees a biharmonic function. We find the coefficients that
best fit the boundary conditions in the least-squares sense. We write
\begin{equation}
f(z) = \sum_{k=1}^N f_j \phi_j(z), \quad g(z) = \sum_{k=1}^N g_j \phi_j(z),
\end{equation}
where $f_j, g_j\in \mathbb{C}$ add up to $4N$ real degrees of freedom, and correspond to the expansion coeffi- cients with respect to the basis functions
\begin{equation}
\left\{\phi_j(z)\right\}_{j=1}^{N} =  \bigcup_{k=1}^K\left\{\frac{1}{z-\beta_{kn}}\right\}_{n=1}^{N_k} \cup \left\{p_n(z)\right\}_{n=0}^{N_0}.
\end{equation}

For the rational functions, we choose poles $\beta_{kn}$ clustered near the corners $w_k$, as described in \cite{gopal19},
\begin{equation}
\beta_{kn} = w_k + Le^{i\theta_k} e^{-\sigma (\sqrt{N_k}-\sqrt{n})}, \quad k=1,\ldots,K,\quad n=1,\ldots,N_k,
\end{equation}
where $\sigma=4$, $\theta_k$ is the angle formed between the real axis and the bisector at the corner $w_k$, and $L$ is a characteristic length scale associated with $\Omega$. For the polynomial basis functions, $p_n(z)$, we construct the  the discretely-orthogonal ``Vandermonde with Arnoldi'' polynomials from \cite{brubeck19}, which obey an $(n+2)$-term recurrence relation
\begin{equation} \label{eq:arnoldi}
H_{n,n+1} p_{n+1}(z) = z p_n(z) - \sum_{k=0}^n H_{n,k} p_k(z),\quad n=1,\ldots,N_0, \quad p_0(z)=1.
\end{equation}
Here $H$ is the upper-Hessenberg matrix from the Arnoldi iteration on the Krylov subspace $\mathcal{K}_{N_0+1}(\diag(z_i), e)$, where $e$ is the vector with all entries equal to 1. We can differentiate \eqref{eq:arnoldi} to obtain a recurrence relation for $p_n'(z)$. This allows us to treat boundary conditions involving $f'(z)$ and $g'(z)$.



\subsection{Linear least-squares system}
The two boundary conditions \eqref{eq:bcs} are enforced on a set of $M > 2N$ sample points $\{z_i \}_{i=1}^M$, which
are clustered near the corners similarly as $\beta_{kn}$, by approximately solving via least-squares
\begin{equation} \label{eq:LS}
A x\approx b,
\end{equation}
where $A\in\reals^{2M\times 4N}$, $x\in\reals^{4N}$ and $b\in\reals^{2M}$. If we partition the system into blocks, each associating the complex coefficients for a basis function $\phi_j(z)$ with the point $z=z_i$ where it or its derivative is evaluated, we have
\begin{equation}
\matlabmatrix{A_{11} {\cdots} A_{1N}; {\vdots} {\ddots} {\vdots}; A_{M1} {\cdots} A_{MN}}
\matlabmatrix{x_{1}; {\vdots}; x_{N}} \approx \matlabmatrix{b_{1}; {\vdots}; b_{M}},
\end{equation}
where $A_{ij}\in\reals^{2\times 4}$ are the blocks associated with the basis function $\phi_{j}(z)$ or its derivative evaluated at $z=z_i$, $b_i\in\reals^2$ correspond to the RHS for the two BCs imposed at $z=z_i$, and $x_j\in\reals^4$ contain the 4 real degrees of freedom of the complex coefficients of $\phi_j(z)$ in $f(z)$ and $g(z)$,
\begin{equation}
x_ j = \matlabmatrix{\real{f_j} \imag{f_j} \real{g_j} \imag{g_j}}^\transp.
\end{equation}


For instance, a condition on both velocity components imposed at $z=z_i$ will correspond to two rows with right-hand side entries $b_i=(U(z_i), -V(z_i))^\transp$ and blocks
\begin{equation}
\renewcommand*{\arraystretch}{1.5}
A_{ij}=\matlabmatrix{\real{\conj{z_i}\phi'_j(z_i)-\phi_j(z_i)} -\imag{\conj{z_i}\phi'_j(z_i)-\phi_j(z_i)} \real{\phi'_j(z_i)} -\imag{\phi'_j(z_i)}; 
\imag{\conj{z_i}\phi'_j(z_i)+\phi_j(z_i)} \phantom{+}\real{\conj{z_i}\phi'_j(z_i)+\phi_j(z_i)} \imag{\phi'_j(z_i)} \phantom{+}\real{\phi'_j(z_i)}},
\end{equation}
for $j=1,\ldots,N$. A no-slip condition can be implemented by fixing $b_i=(0,0)^\transp$.
For a slip condition, one may also specify the tangential velocity component and set the stream function to some constant value $\psi_0$ along the boundary. At the point $z=z_i$, this will correspond to two rows with RHS $b_i=(U_t(z_i),\psi_0)^\transp$ and blocks
\begin{equation}
\renewcommand*{\arraystretch}{1.5}
A_{ij}=\matlabmatrix{\imag{F_{ij}} \real{F_{ij}} \imag{G_{ij}} \real{G_{ij}}; 
\imag{\conj{z_i}\phi_j(z_i)} \real{\conj{z_i}\phi_j(z_i)} \imag{\phi_j(z_i)} \real{\phi_j(z_i)}},
\end{equation}
for $j=1,\ldots,N$, where $F_{ij}=e^{i\alpha_i}\conj{z_i}\phi'_j(z_i)+e^{-i\alpha_i}\phi_j(z_i)$,  $G_{ij}=e^{i\alpha_i}\phi'_j(z_i)$, and $\alpha_i$ denotes the angle between the vector normal to the boundary and the real axis at the point $z=z_i$.

In our implementation, we solve \eqref{eq:LS} via the MATLAB backslash command. In order to get good results, it is crucial to do column scaling, which we do by right multiplying by
\begin{equation}
D=\diag( A^\transp A)^{-1/2}.
\end{equation} 
This can be illustrated in a few lines of code,
\begin{verbatim}
D = spdiag(1./sqrt(sum(A.^2,1)));
y = (A*D)\b; 
x = D*y;
\end{verbatim}


\subsection{Adaptive strategy}
We use a similar adaptive strategy to the one that is presented in \cite{gopal19}, which consists of constructing a sequence of linear least-squares problems of increasing size. In brief, we associate each entry of the residual vector $r=b-Ax$ with the nearest corner $w_k$ by defining the index sets
\begin{equation}
\mathcal{I}_k = \Big\{1\leq i\leq M : k=\arg\min_{1\leq l \leq K} \abs{w_l-z_i}\Big\},\quad k=1,\ldots, K.
\end{equation}
After solving the least-squares problem, we decide to increase the number of poles based on the partial residuals, 
\begin{equation}
R_k^2 = \sum_{j\in \mathcal{I}_k} r_j^2,\quad k=1,\ldots, K,
\end{equation}
on which we do a histogram. In the next least-squares problem, we increase the number of poles at every corner, but more poles will be placed near the corners that fall in the bin with the largest residual. The computation is terminated if the residual estimated on a finer boundary mesh is sufficiently small or $N$ is too large or the error is not decreasing.


\subsection{The algorithm}
Here is a summary of our algorithm. 

\begin{algorithm}[H]
	\SetAlgoLined
	Define boundary $\partial \Omega$, corners $w_1,\ldots,w_K$, boundary functions $h,k,\vec{g}$ and tolerance $\varepsilon$;\\
	\Repeat{Exit criteria have been met}{
		Define rational function basis;\\
		Choose $M$ sampling points $z_1\ldots z_M$, clustered near corners;\\
		Form the matrix $A$ and RHS vector $b$;\\
		Solve the least-squares problem $Ax\approx b$ for the coefficient vector $x$;\\
		Exit loop if $\norm{Ax-b}_2<\varepsilon$ or if $N$ too large or if the error is growing;\\
		Form the residual histogram and increment the number of poles per corner accordingly;\\
	}
	Confirm accuracy by checking error on a finer boundary mesh;\\
	Construct a function to evaluate the solution based on the computed coefficients $x$;\\
	\caption{The lightning Stokes solver.}
\end{algorithm}

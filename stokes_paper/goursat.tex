%!TEX root = stokes_paper.tex
\newtheorem{theorem}{Theorem}

\section{Biharmonic functions in simply-connected domains \label{sec:goursat}}

The Goursat representation is the key ingredient required to extend the lightning solver to biharmonic problems, going back to the French mathematician Goursat in 1898 \cite{goursat}. It allows one to represent biharmonic functions in terms of two analytic functions, which our algorithms will then approximate by rational functions. In line with the use of analytic functions, and throughout the rest of the paper, we represent $x$ and $y$ via the complex variable $z = x + iy$. We include a precise statement of the Goursat representation in the form of a theorem, along with two proofs, which we consider to be valuable in giving two perspectives on what is going on. The original proof was given by Goursat in \cite{goursat}, and an alternative proof is given by Muskhelishvili in \cite{mus19}. Both of these are also found in textbooks such as \cite{ockendon95} and \cite{mus77}. Although we label them ``proofs'', we don’t give these arguments in full detail. For a rigorous discussion, see [\textbf{Find a reference for this}].


\begin{theorem}[Goursat representation]
\label{th:goursat}
Let $\Omega\subseteq\mathbb{C}$ be a simply-connected open region, and let
$\psi : \Omega \rightarrow \reals$ be a biharmonic function. Then $\psi$ can be represented in terms of functions $f(z)$ and $g(z)$ analytic in $\Omega$ by the formula
\begin{equation}\label{eq:goursat}
\psi(x,y) = \psi(z,\conj{z}) = \Im \left\{ \conj{z}f(z) + g(z)\right\}.
\end{equation}
The functions $f(z)$ and $g(z)$ are uniquely defined up to addition of arbitrary terms $\lambda z+C$ to $f(z)$ and $\conj{C}z+\alpha$ to $g(z)$, where $\lambda,\alpha\in\reals$ and $C\in\mathbb{C}$.
\end{theorem}

The use of the imaginary as opposed to the real part is standard in the literature and helps
to simplify the expressions for physical quantities associated with $\psi$. Series expansions based on
\eqref{eq:goursat} have been used to find numerical solutions to BVPs associated with \eqref{eq:bih} by Gaskell et al. in
\cite{gaskell}, Finn, Cox and Byrne in \cite{finn}, and by Luca and Crowdy in \cite{luca18}, among many others. Though
the Goursat representation has been used by various people over the years, it is used far less than more general tools such as FEM and integral equations and indeed is surprisingly hard to find in textbooks.


\begin{proof}[Goursat's proof \cite{Gou98}]
First, we note that the biharmonic operator can be written in terms of Wirtinger derivatives,
\begin{equation}
\nabla^4 \psi = 16\frac{\partial^4 \psi}{\partial\conj{z}^2 \partial z^2}.
\end{equation}
Treating $z$ and $\conj{z}$ as independent variables, one can solve this equation by separation of variables. Under the assumption that $\psi$ is analytic with respect to te real variables $x$ and $y$, we may plug in the ansatz $\psi=\sum_k \psi_k$, where $\psi_k= A_k(z)B_k(\conj{z})$, obtaining
\begin{equation}
\frac{\partial^4 \psi_k}{\partial\conj{z}^2 \partial z^2} = A_k''(z) B_k''(\conj{z}) = 0,
\end{equation} 
which implies that either $A_k''(z)=0$ or $B_k''(\conj{z})=0$. The solution spaces for these two cases are spanned by $\{1,z\}$ for $A_k''(z)=0$  and $\{1,\conj{z}\}$ for $B_k''(\conj{z})=0$. Therefore, without loss of generality, by superposition of the four independent solutions,
\begin{equation}
\psi(z,\conj{z}) =  A_1(z) + \conj{z} A_2(z) + B_3(\conj{z}) + z B_4(\conj{z}),
\end{equation}
where $A_1(z), A_2(z), B_3(\conj{z}), B_4(\conj{z})$ are arbitrary functions. Since $\psi$ must be real, $\psi=\conj{\psi}$ implies that 
\begin{equation}
\psi(z,\conj{z}) =  A_1(z)  + \conj{z} A_2(z)  + \conj{A_1(z)} + z \conj{A_2(z)} = 2 \Re\left\{A_1(z) + \conj{z} A_2(z)\right\},
\end{equation}
which implies \eqref{eq:goursat} by letting $f(z) = -2i A_2(z)$ and $g(z) = -2i A_1(z)$. We finally note that, when $\psi$ is given, $f(z)$ and $g(z)$ are determined up to an additive linear term. To be more precise, the substitutions $f(z)\to f(z)+\lambda z + C$ together with $g(z)\to g(z)+\conj{C}z+\alpha$ for $\lambda,\alpha\in\reals$ and $C\in\mathbb{C}$ leave $\psi$ invariant.
\end{proof}


\begin{proof}[Muskhelishvili's proof \cite{Mus19}]
If $\omega=-\nabla^2 \psi$, then $-\nabla^2\omega = \nabla^4\psi = 0$. Now define $p$ as the harmonic conjugate of $\omega$, satisfying the Cauchy-Riemann conditions
\begin{equation}
\frac{\partial\omega}{\partial x} = \frac{\partial p}{\partial y}, \quad
\frac{\partial\omega}{\partial y} = -\frac{\partial p}{\partial x};
\end{equation}
this function is determined up to an arbitrary constant, $p_0\in\reals$. Then the expression
\begin{equation}
h(z) = p(x,y)-i\omega(x,y)
\end{equation}
represents a function of the complex variable $z=x+iy$, analytic in $\Omega$. Put
\begin{equation}
f(z) := \frac{1}{4} \int_a^z h(w) dw = f_1 + if_2,
\end{equation}
where $a\in\Omega$ is arbitrary, causing $f(z)$ to be defined up to a term $\lambda z +C$ where $4\lambda=p_0$ and $C\in\mathbb{C}$. Obviously,
\begin{equation}\label{eq:fprime}
f'(z) = \frac{\partial f_1}{\partial x} + i \frac{\partial f_2}{\partial x} = -i \frac{\partial f_1}{\partial y} + \frac{\partial f_2}{\partial y} = \frac{1}{4} \left(p-i\omega\right),
\end{equation}
from which, we calculate, using the fact that $\nabla^2 f_1=\nabla^2 f_2=0$,
\begin{equation}
\nabla^2\left(y f_1\right) = 2\frac{\partial f_1}{\partial y} = \frac{\omega}{2},\quad \nabla^2\left(x f_2\right) = 2\frac{\partial f_2}{\partial x} = -\frac{\omega}{2}.
\end{equation}
Thus, the function $\psi + yf_1 - xf_2$ is harmonic, since
\begin{equation}
\nabla^2 \left(\psi + yf_1 - xf_2\right) = -\omega + \frac{\omega}{2} + \frac{\omega}{2} =0.
\end{equation}
Let this function be $\psi + yf_1 - xf_2 = \Im\left\{g(z)\right\}$, where $g(z)$ is analytic, and for the given $f(z)$, it will be defined up to a term $\conj{C}z+\alpha$, where $\alpha\in\reals$, since $\psi$ must not depend on the term $C\conj{z}$. Hence one can represent
\begin{equation}
\psi = x f_2(z) -y f_1(z) + \Im\left\{g(z)\right\} = \Im\left\{\conj{z}f(z) + g(z)\right\}.
\end{equation}
This argument can be used to establish the analyticity of $\psi$.
\end{proof}